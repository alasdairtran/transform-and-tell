% !TEX root = main.tex


\section{Model Architecture}

\begin{figure*}[t]
   \begin{center}
   \fbox{\rule{0pt}{2in} \rule{.9\linewidth}{0pt}}
   \end{center}
      \caption{Overall architecture of the model.}
   \label{fig:short}
\end{figure*}


\subsection{Model Inputs}

Our proposed model takes three inputs: the image, the faces, and article text.
Each of these inputs first goes through an encoder to give us a vector
representation.

\subsubsection{Image Embedding}

For the image, we feed it through a pretrained ResNet-152 model
\cite{He2016ResNet} and use the output of the final block, just before the
pooling layer, as the image embedding $\bm{x}_I \in \mathbb{R}^{2048 \times 7
\times 7}$. The embedding forms a 7 by 7 block, allowing the transformer to
attend to 49 different patches in the image.

\subsubsection{Article Embedding}

For the article, we use RoBERTa \cite{Liu2019RoBERTaAR} to encode the text.
RoBERTa is a language representation model that provides pretrained contextual
embeddings for text. It consists of 24 layers of bidirectional transformer
blocks. It is a more carefully trained BERT \cite{Devlin2019BERT} model.

Unlike GloVe \cite{Pennington2014Glove} and word2vec
\cite{Mikolov2013DistributedRO} embeddings, where each word has exactly one
representation, the bidirectionality and the attention mechanism in the
transformer allow a word to have different vector representations depending on
the surrounding context.

The largest GloVe model has a vocabulary size of 1.2 million. Although this is
large, many rare names will still get mapped to the unknown token. In contrast,
RoBERTa uses BPE \cite{Sennrich2015NeuralMT,Radford2019LanguageMA} which can
encode any word that can be written in Unicode characters.

Given an input of length $S$, the pretrained RoBERTa encoder will return 25
sequences of embeddings, $\bm{h}_i \in \mathbb{R}^{2048 \times S}$ for $i \in
\{0,1, 2,...,24\}$. This includes the initial uncontextualized embeddings and
the output of each of the 24 layers. Inspired by Tenney \etal
\cite{Tenney2019BertRT}, who showed that different layers in BERT represent
different steps in the traditional NLP pipeline, we take a weighted sum
across all layers to obtain the article embedding:
\begin{IEEEeqnarray*}{lCl}
   \bm{x}_T &=& \sum_{i=0}^{24} w_i \bm{h}_i
\end{IEEEeqnarray*}
where $\bm{x}_T \in \mathbb{R}^{1024 \times S}$ is the article embedding
and $w_i$ are learnable weights.

One limitation of RoBERTa is that the maximum length of the input sequence is
512. For GoodNews, we simply encode the first 512 tokens of the article. For
NYTimes800k, since we have the image position, we concatenate the title, the
first paragraph, and as many paragraphs above and below the image as we can
fit, until we reach the 512 token limit. Note that since we are using BPE, a
token can be part of a word. On average, we can only encode ...... words of
the article.


\subsubsection{Face Embedding}

To embed the faces, we first use MTCNN \cite{Zhang2016JointFD} to detect the
face bounding boxes. We then select the top $M$ faces and feed them through a
FaceNet model \cite{Schroff2015FaceNetAU}, pretrained on the VGGFace2 dataset
\cite{Cao2017VGGFace2AD}, to obtain a face embedding $\bm{x}_F \in
\mathbb{R}^{512 \times M}$ where $M$ is the number of faces.

\subsection{Decoder}

\subsubsection{Dynamic Convolutions}

Our decoder is a transformer architecture with dynamic convolutions
\cite{Wu2018PayLA} containing four layers with kernel sizes 3, 7, 15 and 31. We
found that using dynamic convolutions allows the training to converge faster
than using the current state-of-the-art GPT-2 architecture
\cite{Radford2019LanguageMA}.

In the standard self-attention module, the input embedding $\bm{x} \in
\mathbb{R}^{1024}$ is first projected linearly into a query, key, and value
$\bm{q}, \bm{k}, \bm{v} \in \mathbb{R}^{1024}$. Then the attention is the
weighted sum of the values, where the weights are obtained from the dot product
between the key and value:
\begin{IEEEeqnarray*}{lCl}
   \text{Attention}(\bm{x}) &=&
   \text{softmax}\left( \dfrac{\bm{q} \bm{k}^T}{\sqrt{d_K}} \right) \bm{v}
\end{IEEEeqnarray*}

\subsubsection{Adaptive Softmax}

The BPE vocabulary size is 50265. To make training more efficient, we use adaptive softmax
\cite{Grave2016EfficientSA} and divide the vocabulary into three clusters: 5K,
15K, and 25K. We tie the adaptive weights and we share the decoder input and
output embeddings. We use sinusoidal positional encoding
\cite{Vaswani2017AttentionIA} to represent the position of each token.

\subsubsection{Encoder Attention}

For each generation step, we use the Scalar-Dot-Product attention
\cite{Vaswani2017AttentionIA} to attend to the image, article, and faces. All
of the three attended representations are then concatenated and fed as input to
the self-attention layer.
